\documentclass{article}
\usepackage{amsmath} % Diperlukan untuk simbol matematika
\usepackage{graphicx}
\usepackage{venndiagram}
\usepackage{float}
\usepackage{ragged2e}
\begin{document}
\begin{justify}
    Ahmad Faiq Fawwaz\\5024231032\\
    \end{justify} 

1. Hukum Idempoten
\begin{align*}
A \cup A &= A 
\end{align*}
%\begin{venndiagram3sets}
%\fillAcupB
%\end{venndiagram3sets}
\begin{figure}[H]
    \centering
\end{figure} 

\begin{align*}
    A \cap A &= A
\end{align*}

\begin{figure}[H]
    \centering
\end{figure} 

2. Hukum Komutatif
\begin{align*}
A \cup B &= B \cup A\\
\end{align*}
\begin{figure}[H]
    \centering
\end{figure} 

\begin{align*}
  A \cap B &= B \cap A  
\end{align*}
\begin{figure}[H]
    \centering
\end{figure} 



3. Hukum Asosiatif
\begin{align*}
(A \cup B) \cup C &= A \cup (B \cup C) 
\end{align*}
\begin{figure}[H]
    \centering
\end{figure} 
\begin{align*}
(A \cap B) \cap C &= A \cap (B \cap C)
\end{align*}
\begin{figure}[H]
    \centering
\end{figure} 

4. Hukum Distributif
\begin{align*}
A \cup (B \cap C) &= (A \cup B) \cap (A \cup C) 
\end{align*}

\begin{figure}[H]
    \centering
\end{figure} 

\begin{align*}
A \cap (B \cup C) &= (A \cap B) \cup (A \cap C)
\end{align*}
\begin{figure}[H]
    \centering
\end{figure} 

5. Hukum Komplemen
\begin{align*}
A \cup A^c &= U \quad \text{(dimana } U \text{ adalah himpunan semesta)} 
\end{align*}


\begin{figure}[H]
    \centering
\end{figure} 

\begin{align*}
A \cap A^c &= \emptyset \quad \text{(dimana } \emptyset \text{ adalah himpunan kosong)}
\end{align*}
    
\begin{figure}[H]
    \centering
\end{figure} 

6. Hukum Identitas
\begin{align*}
A \cup \emptyset &= A 
\end{align*}

\begin{figure}[H]
    \centering
\end{figure} 

\begin{align*}
A \cap U &= A
\end{align*}

\begin{figure}[H]
    \centering
\end{figure} 

7. Hukum Dominasi
\begin{align*}
A \cup U &= U 
\end{align*}
\begin{figure}[H]
    \centering
\end{figure} 
\begin{align*}
A \cap \emptyset &= \emptyset
\end{align*}
\begin{figure}[H]
    \centering
\end{figure} 

8. Hukum Absorpsi
\begin{align*}
A \cup (A \cap B) &= A 
\end{align*}
\begin{figure}[H]
    \centering
\end{figure} 
\begin{align*}
A \cap (A \cup B) &= A
\end{align*}
\begin{figure}[H]
    \centering
\end{figure} 

9. Hukum Komplemen
\[
(A^c)^c = A
\]
\begin{figure}[H]
    \centering
\end{figure} 
10. Hukum De Morgan
\begin{align*}
(A \cup B)^c &= A^c \cap B^c 
\end{align*}
\begin{figure}[H]
    \centering
\end{figure} 
\begin{align*}
(A \cap B)^c &= A^c \cup B^c
\end{align*}
\begin{figure}[H]
    \centering
\end{figure} 
\end{document}
